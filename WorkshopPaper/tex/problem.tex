\section{Motivating Problem}

As an example of a data mining task, we consider the problem of predicting the mode of transport of a mobile user. Mode of transport prediction is an interesting problem in data mining. It helps in providing contextual information about the user that can be used in building intelligent smartphone applications.

Reddy et al. \cite{MTP} describe a method for predicting the mode of transport of a mobile user by applying the Decision Trees algorithm with features as the Discrete Fourier Transform (DFT) energy coefficients at 1Hz, 2Hz, and 3Hz of acceleration magnitude, the accelerometer variance, and the speed recorded by the GPS sensor. The features are derived from data collected from the accelerometer and the GPS sensors on a mobile device. Our method of predicting the mode of transport is similar to theirs as we use the Support Vector Machine (SVM) algorithm with similar features. We use GPS accuracy as a feature instead of accelerometer variance, keeping the rest of the features the same.

To acquire training and testing data, we use an application that collects accelerometer and GPS readings on mobile phones. We asked users to record their mode of transportation while using the application in order to acquire the data to build the prediction model. The data mining system derives the aforementioned features from the sensor data and merges the mode of transport labels noted by the user with the features, as classes, for mining.

The collected data is hardly ever ready to use directly in real life scenarios. We have to prepare it for mining through multiple transformation steps, such as dealing with missing values, discretization, and normalization. Moreover, even if the data is in tabular form, in many cases, we cannot use the columns as features for data mining libraries directly. We have to compute useful features by using values in multiple columns. In the example case of mode of transport prediction, we calculate acceleration magnitude from the acceleration recorded by the accelerometer along the X, Y, and Z dimensions. We then calculate DFT energy coefficients at 1Hz, 2Hz, and 3Hz of the acceleration magnitude for every 1 second time window. We then discretize the data, which contains multiple entries every second, to the nearest second. 

Many data mining tasks use data from multiple sources that have to be merged in order to derive features from them, such as those performed by large organizations on data collected from different branches and departments within them. The data used by the algorithm that we employ in our example case is derived from three sources, i.e., the accelerometer sensor, the GPS sensor, and the user’s mode of transportation notes. We merge the data by making use of timestamps contained in all the three sources.

In general, after the data is prepared, machine learning algorithms are applied to it to train the prediction model and to test it, or to perform cross-validation. In our case, we train a prediction model using the SVM algorithm and test it. The final set of features consists of DFT energy coefficients at 1Hz, 2Hz, and 3Hz, speed recorded by the GPS sensor, and GPS accuracy.

These parts of the entire data mining system motivate the development of a complete end-to-end system that offers complex data transformations (as in-built features, extensions, or services), data merging capabilities, invocation of various machine learning algorithms for application and comparison, and display of results in a meaningful manner.